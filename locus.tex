\documentclass[11pt,a4paper]{ltjsarticle}
\usepackage{luatexja-fontspec}
\usepackage{amsthm}
\usepackage{enumitem}
\usepackage{sty/nogutaku}
\usepackage[colorlinks=true,unicode,linkbordercolor={0 0 0}]{hyperref}
\usepackage{cleveref}
\usepackage{authblk}
\usepackage[margin=25mm,truedimen]{geometry}
%
%
%
%
\setfraction
%
%
\newcommand*{\definition}[1]{\textbf{#1}}
\newcommand*{\length}[1]{\overline{#1}}
\newcommand*{\equivalent}{\Longleftrightarrow}
\newcommand*{\coord}[1]{\paren{#1}}
%
%
%
\crefname{figure}{図}{図}
\crefname{equation}{式}{式}
\crefname{que}{問}{問}
\crefname{ans}{解答}{解答}
%
%
\theoremstyle{definition}
%
%
\setlist[1]{
labelindent = \parindent
}
\setlist[enumerate,1]{
  label = {(\arabic*)},
  ref = \arabic*
}
%
\newtheorem{dfn}{定義}[section]
\newtheorem{thm}[dfn]{定理}
\newtheorem{ex}[dfn]{例}
\newtheorem{que}[dfn]{問題}
\newtheorem{ans}[dfn]{解答}
\newtheorem{lemma}[dfn]{補題}
\newtheorem{coro}[dfn]{系}

%
\title{高校数学Ⅱの軌跡を完全に理解するための$n$問}
\author{野口匠}
\affil{高知工科大学大学院工学研究科}
\date{\today}

\begin{document}
%
\maketitle

この記事はMath Advent Calendar 2019 (\url{https://adventar.org/calendars/4297}) の19日目の記事です.

\section{準備} \label{sec:pre}

この記事では,以下のような表記法を採用する.
なお,高校の教科書における流儀とは異なる部分が多数見られるので注意せよ.

\begin{itemize}
  \item 点を表す記号は高校の教科書ではローマン体で$\mathrm{A}, \mathrm{B}, \dotsc$のように
    書かれることが多いが,この記事では通常の変数と同じく$A,B, \dotsc$のようにイタリック体で表記する.
  \item なんらかの対象に関する条件$\varphi$に対し,対象$x$が条件$\varphi$を満たすという主張を$\apply{\varphi}{x}$と表す.
  \item 集合$S$と条件$\varphi$に対し,$\varphi$を満たす$S$の元全体の集合を$\setintension{x \in S}{\apply{\varphi}{x}}$と表す.
  \item 平面上の点$A,B$に対し,線分$AB$の長さを$\length{AB}$と表す.$\length{AB}$は$A,B$の距離に相当する.
  \item 実数全体の集合を$\Real$で,(ユークリッド)平面を$\Real ^2$でそれぞれ表す.
    $x \in \Real$は$x$が実数であることを,$P \in \Real ^2$は$P$が平面上の点であることをそれぞれ表す.
  \item 「座標が$\coord{a,b}$であるような点$P$」を高校の教科書ではよく「点$P{\coord{a,b}}$」などと表すようであるが,
    この記事では「$P = \coord{a,b}$」と表すこととする.筆者個人としては「$P{\coord{a,b}}$」なる表記は意味がよくわからないので好きではない.
  \item 2つの主張$\varphi, \psi$について,$\varphi$が真であるならば必ず$\psi$も真であること,
    および$\varphi$と$\psi$が同値であることをそれぞれ$\varphi \implies \psi$と$\varphi \equivalent \psi$で表す.
  \item $x < y$または$x = y$であることを表す不等号として$x \leq y$を用いる.
  \item この記事では,特に断らない限り「平面上の点」を単に「点」と表すことがある.
\end{itemize}


\section{軌跡の定義と例} \label{sec:definition}



数学では,「ある対象がどのような性質をもつか」だけでなく,「ある性質をもつものにはどのようなものがあるか」
について考えることも多い.
ここでは,「ある性質をもつ平面上の点全体がどのような図形をなすか」ということについて考えよう.
以下のように定義しておくと便利である.

\begin{dfn} \label{dfn:locus}
  平面$\Real ^2$上の点に関する条件$\varphi$が与えられたとする.
  このとき,$\varphi$を満たす点全体の集合,すなわち,集合
  \begin{align}
    \setintension{P \in \Real ^2}{\apply{\varphi}{P}}
    \label{eq:locus}
  \end{align}
  を(平面上における)$\varphi$を満たす点の\definition{軌跡}%
  \footnote{%
    「$\apply{\varphi}{P}$を満たす点$P$の軌跡」などと表記することもある.%
  }%
  という.
\end{dfn}

\begin{ex} \label{ex:circle}
  正の実数$r$と点$C$が与えられたとき,$\length{CP} = r$を満たす点全体の集合は$C$を中心とする半径$r$の円である.
  このことは「$\length{CP} = r$を満たす点の軌跡は$C$を中心とする半径$r$の円である」と表記される.
\end{ex}

\section{軌跡の求め方}



\Cref{ex:circle}では,その軌跡がすぐにわかるような条件が与えられた.以降はそうでない場合について考える.
ここからは問題形式で話を進めることとしよう.

\begin{que} \label{que:apollonius}
  平面上に2点$A = \coord{-2, 0}$と$B = \coord{4, 0}$がある.$\length{AP} : \length{BP} = 2 : 1$を満たす
  点$P$の軌跡を求めよ.
\end{que}

\begin{ans} \label{ans:apollonius}
  $\length{AP} : \length{BP} = 2 : 1$を満たす平面上の任意の点$P = \coord{x,y}$に対し,
  $\length{AP}^2 = 4 \length{BP}^2$だから
  \begin{align*}
    \paren{x + 2}^2 + y^2 & = 4\paren{\paren{x - 4}^2 + y^2
  }\end{align*}
  より
  \begin{align}
    \paren{x - 6}^2 + y^2 = 16
    \label{eq:apollonius}
  \end{align}
  が成り立つ.
  また,\cref{eq:apollonius}を満たす任意の点$P = \coord{x,y}$は上の計算を逆にたどることによって
  $\length{AP} : \length{BP} = 2 : 1$を満たすことがわかる.
  ゆえに$\length{AP} : \length{BP} = 2 : 1$を満たす点$P$の軌跡は
  点$\coord{6,0}$を中心とする半径4の円である.
\end{ans}

\Cref{que:apollonius}にあるような条件が与えられたとしても,その条件を満たす点の軌跡がどのようなものであるかは
すぐにはわからない.そこでまずは$\length{AP} : \length{BP} = 2 : 1$であると仮定したら何が起こるかを考えるという
指針で議論を進めたのが\cref{ans:apollonius}である.
その結果,$\length{AP} : \length{BP} = 2 : 1$を満たすような点$P = \coord{x,y}$は
必ず$\paren{x - 6} + y^2 = 16$を満たさなければならないということがわかったのであった.
これは,平面上の任意の点$P = \coord{x,y}$が
\begin{align}
  \length{AP} : \length{BP} = 2 : 1 \implies \paren{x - 6}^2 + y^2 = 16
  \label{eq:apolloniusimplies}
\end{align}
を満たすことに相当する.
このことから,
\begin{align}
  \setintension{P \in \Real ^2}{\length{AP} : \length{BP} = 2 : 1} \subset
  \setintension{\coord{x,y} \in \Real^2}{\paren{x - 6}^2 + y^2 = 16}
  \label{eq:apolloniussubset}
\end{align}
であることがわかる.
次に考えなければならないのは,\cref{eq:apolloniussubset}を逆にした
\begin{align}
  \setintension{\coord{x,y} \in \Real^2}{\paren{x - 6}^2 + y^2 = 16} \subset
  \setintension{P \in \Real ^2}{\length{AP} : \length{BP} = 2 : 1}
  \label{eq:apolloniussubsetinv}
\end{align}
が成り立つかどうかである.
\Cref{eq:apolloniussubsetinv}を示すためには
平面上の任意の点$P = \coord{x,y}$に対して
\begin{align}
  \paren{x - 6}^2 + y^2 = 16 \implies \length{AP} : \length{BP} = 2 : 1
  \label{eq:aplloniusimpliesinv}
\end{align}
を示せばよい.すなわち,$\paren{x - 6}^2 + y^2 = 16$である点$P = \coord{x, y}$が
つねに$\length{AP} : \length{BP} = 2 : 1$を満たすことを示せばよい.
これは\cref{eq:apolloniussubset}を示すときに
行った計算をそのまま逆にたどればすぐにわかる.
以上をもって
\begin{align}
  \setintension{\coord{x,y} \in \Real^2}{\paren{x - 6}^2 + y^2 = 16} =
  \setintension{P \in \Real ^2}{\length{AP} : \length{BP} = 2 : 1}
  \label{eq:apolloniusequal}
\end{align}
が成り立つことがわかったので,$\length{AP} : \length{BP} = 2 : 1$を満たす
点$P$の軌跡は点$\coord{6, 0}$を中心とする半径$4$の円であることが結論されるのである.


\Cref{ans:apollonius}では,\cref{eq:aplloniusimpliesinv}を示すのは
\cref{eq:apolloniusimplies}の導出で行った計算過程を逆にたどるだけで終わったのであった.
このことに着目すれば,次のようにして解答を大幅に省略できる.

\begin{ans} \label{ans:aplloniusomit}
  平面上の任意の点$P = \coord{x,y}$に対し,
  \begin{align*}
    & \length{AP} : \length{BP} = 2 : 1 \\
    \equivalent & \length{AP}^2 = 4 \length{BP}^2 \\
    \equivalent & \paren{x + 2}^2 + y^2 = 4 \paren{\paren{x - 4}^2 + y^2} \\
    \equivalent & \paren{x - 6}^2 + y^2 = 16
  \end{align*}
  であるから,$\length{AP} : \length{BP} = 2 : 1$を満たす点$P$の軌跡は
  点$\coord{6, 0}$を中心とする半径$4$の円である.
\end{ans}


直接的に軌跡とは書いてはいないが,実質的に軌跡を求めることに相当するような問題についても考えよう.

\begin{que} \label{que:vertbisector}
  原点$O$と点$A = \coord{2, 2}$に対し,線分$OA$の垂直二等分線の方程式を求めよ.
\end{que}

\begin{ans} \label{ans:vertbisector}
  線分$OA$の垂直二等分線上の任意の点$P = \coord{x, y}$に対し,
  $\length{OP} = \length{AP}$が成り立つから$\length{OP}^2 = \length{AP}^2$
  より
  \begin{align}
    x + y - 2 = 0
    \label{eq:vertbisector}
  \end{align}
  が成り立つ.
  また,\cref{eq:vertbisector}を満たす任意の点$P = \coord{x, y}$に対し,
  上の計算を逆にたどることによって$\length{OP} = \length{AP}$が成り立つことがわかる.
  ゆえに,$P$は線分$OA$の中点であるか,さもなくば$\triangle OAP$が$\length{OP} = \length{OA}$
  の二等辺三角形であるかのいずれかであり,いずれにせよ点$P$は線分$OA$の垂直二等分線上の点である.
  従って線分$OA$の垂直二等分線の方程式は\cref{eq:vertbisector}で与えられる.
\end{ans}

\Cref{ans:vertbisector}の議論の進め方が\cref{ans:apollonius}とまったく同じであることは
いうまでもない%
\footnote{%
  そこそこ有名な大学受験対策サイトに「『軌跡を求めよ』ではなく『方程式を求めよ』なのだから
  逆の確認(\cref{ans:vertbisector}での「または」からの部分)はしなくてもよい」
  などと書いてあったのだが,もちろんそんなことはない.
}%
.
もちろん\cref{ans:aplloniusomit}のような形式の解答も述べることができるが,いちいち書く必要はないだろう.

もう少し状況を複雑にして考えてみよう.このあたりから市販の参考書に間違いが多くなってくる.

\begin{que} \label{que:varpoint}
  点$\coord{2,0}$を中心とする半径2の円$C$と点$A = \coord{4,0}$がある.
  点$Q$が3点$O, A, Q$が一直線上にない
  \footnote{%
    ここを書いていない解説がほとんどなのであるが,
    書かないとかなり嫌がらせチックな問題文になってしまうので明示しておくこととする.
  }%
  ような$C$上の点全体を動くとき,
  $\triangle OAQ$の重心の軌跡を求めよ.ただし$O$は原点である.
\end{que}

\begin{ans} \label{ans:varpoint}
  $C$と直線$OA$の交点は2点$O, A$である.
  これ以外の$C$上の任意の点$Q = \coord{s, t}$に対し,
  \begin{align}
    \paren{s - 2}^2 + t^2 = 4
    \label{eq:varcond}
  \end{align}
  が成り立つ.
  いま,$\triangle OAQ$の重心を$P = \coord{x, y}$とおくと,
  \begin{align}
    \begin{aligned}
      x & = \frac{s + 4}{3}, \\
      y & = \frac{t}{3}
    \end{aligned}
    \label{eq:varpoint}
  \end{align}
  であるから
  \begin{align}
    \begin{aligned}
      s & = 3x - 4, \\
      t & = 3y
    \end{aligned}
    \label{eq:varpointinv}
  \end{align}
  であり,これを\cref{eq:varcond}に代入して
  \begin{align}
    \paren{x - 2}^2 + y^2 = \frac{4}{9}
    \label{eq:varpointlocus}
  \end{align}
  を得る.また,$Q$は$O,A$のいずれとも異なる点であるから,
  \cref{eq:varpoint}より$P$は2点$\coord{\frac{4}{3}, 0}, \coord{\frac{8}{3}, 0}$
  のいずれとも異ならなければならない.

  逆に,\cref{eq:varpointlocus}を満たす2点$\coord{\frac{4}{3}, 0}, \coord{\frac{8}{3}, 0}$
  のいずれとも異なる任意の点$P = \coord{x,y}$に対し,
  \cref{eq:varpointinv}で定められる点$Q = \coord{s, t}$
  は2点$O, A$のいずれとも異なり,かつ$C$上の点である.
  
  以上より,$Q$が3点$O,A,Q$が一直線上にないような$C$上の点全体を動くとき,
  $\triangle OAQ$の重心の軌跡は点$\coord{2, 0}$を中心とする半径$\frac{4}{9}$
  の円から2点$\coord{\frac{4}{3}, 0}, \coord{\frac{8}{3}, 0}$を除いた部分である.
\end{ans}

\Cref{ans:varpoint}の「逆に」からの部分をサボる解説を非常によく見るのだが,
明確な間違いであることはいうまでもない.

さて,\cref{ans:varpoint}の各部分が何をしているのかを整理しよう.
今回求めたのは,「$P$が$\triangle OAQ$の重心となるような3点$O,A,Q$が一直線上にないような$C$上の点$Q$が存在する」
という条件を満たす点$P$の軌跡である.ゆえに問題文でも「軌跡」という用語を用いている.
ゆえに前半では,上の条件を満たす点$Q$がうまくとれたことを前提にして議論を進めた.
その結果そのような点$P$は\cref{eq:varpointlocus}を満たし,
かつ2点$\coord{\frac{4}{3}, 0}, \coord{\frac{8}{3}, 0}$のいずれとも異ならなければならないということがわかったのであった.
後半では,\cref{eq:varpointlocus}を満たす2点$\coord{\frac{4}{3}, 0}, \coord{\frac{8}{3}, 0}$
のいずれとも異なる点$P$すべてが「$P$が$\triangle OAQ$の重心となるような3点$O,A,Q$が一直線上にないような$C$上の点$Q$が存在する」
という条件を満たすことを確かめるため,そのような点$Q$を実際に構成してみせた.
どのようにして構成すればよいかは前半の議論でほとんど明らかになっているため,
それをトレースするのみである.

なお,\cref{ans:varpoint}を\cref{ans:aplloniusomit}のように前半と後半の議論を同時に行うことは
できなくもないがやたら記述が複雑になるのでオススメしない.



\section{逆像法とパラメータ表示}

最後に,受験業界でよく逆像法などと呼ばれている手法とパラメータ表示に関して少しだけ議論することとする.
逆像法に関しては,筆者に言わせればなぜいちいち名前をつけなければならないのかよくわからない程度には
自然なテクニックである.実は\cref{ans:varpoint}の後半の議論は逆像法に近い議論の進め方をしている.

まずは逆像法から始めよう.何が逆像法かは正直どうでもよいので特に解説はしない.

\begin{que} \label{que:invimage}
  点$\coord{x,y}$が平面上の点全体を動くとき,点$\coord{x+y, xy}$の軌跡を求めよ.
\end{que}

\begin{ans} \label{ans:invimage}
  平面上の任意の点$\coord{x,y}$に対し,$x = X + Y$と$y = XY$をともに満たす点$\coord{X,Y}$が存在するためには,
  $u$の2次方程式$u^2 - xu + y = 0$が実数解をもつこと,すなわち$u$の2次多項式$u^2 - xu + y$の
  判別式$D = x^2 - 4y$が負でないことが必要十分である.これが成り立つのは$y \leq \frac{x^2}{4}$のときである.
  ゆえに点$\coord{x, y}$が平面上の点全体を動くとき,点$\coord{x+y, xy}$の軌跡は
  放物線$y = \frac{x^2}{4}$上の点とその下側の点全体である.
\end{ans}

\Cref{ans:invimage}は\cref{ans:aplloniusomit}と同じ議論の進め方をしている.
これくらい単純な議論で済む場合にはこの型で議論を進めた方が簡潔でわかりやすいであろう.

最後にパラメータ表示を一例だけ扱おう.
パラメータ表示は高校の教科書では媒介変数表示などと書かれているようであるが,
いい加減用語を書き換えるべきだと思う.

\begin{que} \label{que:parameter}
  $t$が実数全体を動くとき,
  \begin{align}
    \begin{aligned}
      x & = \frac{1 - t^2}{1 + t^2}, \\
      y & = \frac{2t}{1 + t^2}
    \end{aligned}
    \label{eq:parameter}
  \end{align}
  を満たす点$\coord{x, y}$の軌跡を求めよ.
\end{que}

\begin{ans} \label{ans:parameter}
  平面上の任意の点$\coord{x, y}$に対し,\cref{eq:parameter}を満たす実数$t$が存在したとする.
  このとき,$x = \frac{\paren*{1 - t^2}}{\paren*{1 + t^2}}$より$x \neq -1$でなければならず,
  さらに$t^2 = \frac{\paren*{1 - x}}{\paren*{1 + x}}$
  である.ゆえに$y = \frac{\paren*{2t}}{\paren*{1 + t^2}}$から
  \begin{align}
    t = \frac{y}{1 + x}
    \label{eq:parametert}
  \end{align}
  を得る.
  これと$t^2 = \frac{\paren*{1 - x}}{\paren*{1 + x}}$より
  \begin{align}
    x^2 + y^2 = 1
    \label{eq:parametercurve}
  \end{align}
  が成り立つ.

  また,$x \neq -1$と\cref{eq:parametercurve}を満たす任意の点$\coord{x, y}$に対し,
  実数$t$を\cref{eq:parametert}によって定めれば,
  $t^2 = \frac{y^2}{\paren*{1 + x}^2} = \frac{\paren*{1 - x^2}}{\paren*{1 + x}^2} 
  = \frac{\paren*{1 - x}}{\paren*{1 + x}}$より$x = \frac{\paren*{1 - t^2}}{\paren*{1 + t^2}}$
  が成り立ち,さらに$y = t \paren{1 + x} = \frac{\paren*{2t}}{\paren*{1 + t^2}}$だから
  \cref{eq:parameter}が成り立っていることがわかる.

  以上の議論により,$t$が実数全体を動くとき,\cref{eq:parameter}を満たす点$\coord{x,y}$
  は円$x^2 + y^2 = 1$から点$\coord{-1, 0}$を除いた部分である.
\end{ans}

この逆の議論をすっぽかしている参考書が非常に多い.
主にある条件を満たす点や実数の存在によって規定される点の軌跡を求める際に
間違える参考書が多いようである.
\end{document}
